\documentclass[11pt,a4paper]{article}

\usepackage[margin=1in]{geometry}
\usepackage{amsmath,amssymb,mathtools,amsthm}
\usepackage{physics}
\usepackage{siunitx}
\usepackage{graphicx}
\usepackage{booktabs}
\usepackage{microtype}
\usepackage{csquotes}
\usepackage{bm}
\usepackage{hyperref}
\usepackage{listings}
\usepackage{xcolor}
\usepackage{fancyhdr}
\usepackage{titlesec}
\usepackage{tcolorbox}

% --- Custom commands ---
\newcommand{\hc}{\hat{a}}
\newcommand{\hcd}{\hat{a}^\dagger}
\newcommand{\sigmap}{\hat{\sigma}^+}
\newcommand{\sigmam}{\hat{\sigma}^-}
\newcommand{\sigmaz}{\hat{\sigma}_z}
\newcommand{\HJC}{\hat{H}_{\text{JC}}}
\newcommand{\Hfull}{\hat{H}_{\text{full}}}
\newcommand{\Hdisp}{\hat{H}_{\text{disp}}}
\newcommand{\OQP}{Operational Quantum Physics}
\newcommand{\POVM}{Positive Operator-Valued Measure (POVM)}

\theoremstyle{definition}
\newtheorem{definition}{Definition}[section]
\newtheorem{theorem}{Theorem}[section]

\hypersetup{
  colorlinks=true,
  linkcolor=blue,
  citecolor=blue,
  urlcolor=blue
}

\sisetup{separate-uncertainty=true}

\pagestyle{fancy}
\fancyhf{}
\rhead{\small \textit{Simulating Open Quantum Systems}}
\lhead{\small \textbf{Theory \& Implementation}}
\cfoot{\thepage}

\title{\textbf{The Jaynes--Cummings Model and Operational Quantum Physics}\\
\large A Comprehensive Theoretical Framework for \texttt{jc-sim-oqp}}
\author{Atom Computing Software Team}
\date{\today}

\begin{document}

\maketitle

\begin{abstract}
The Jaynes--Cummings (JC) model describes the coherent interaction between a single quantized mode of the electromagnetic field and a two-level system. Since its introduction in 1963, it has become a cornerstone of cavity QED and modern quantum information processing. This document serves as the comprehensive theoretical reference for the \texttt{jc-sim-oqp} software package.

It merges the standard pedagogical treatment of the JC model (derivations, dressed states, collapse/revivals) with the rigorous \textbf{Operational Quantum Physics (OQP)} framework derived in the author's early reports. We derive the \textbf{Lindblad Master Equation} from microscopic reservoir theory, justify \textbf{Stochastic Unravellings} (Quantum Trajectories) via the theory of \textbf{Minimal Dilation} (including detailed proofs), and frame \textbf{Dispersive Readout} as a non-ideal \textbf{Quantum Instrument}. Finally, we connect these concepts to the experimental realities of neutral-atom and superconducting circuit platforms, providing guidelines for numerical simulation.
\end{abstract}

\tableofcontents
\newpage

%=============================================================================
\part{The Coherent Jaynes--Cummings Model}
%=============================================================================

\section{Introduction}
%=============================================================================

The Jaynes--Cummings (JC) model is the paradigmatic exactly solvable model of light--matter interaction: it describes a single two-level atom interacting with a single quantized mode of the electromagnetic field.\cite{JaynesCummings1963ProcIEEE}
It may be regarded as the fully quantum counterpart of the semiclassical Rabi model, where the field is treated as a classical drive.
Beyond its pedagogical value, the JC model quantitatively captures many experiments in cavity QED and has inspired a vast literature on quantum control, quantum information, and fundamental tests of quantum mechanics.

At the level of physical implementation, the JC Hamiltonian accurately models the coupling between:
\begin{itemize}
  \item A \textbf{two-level atom} (or effective qubit) with ground state $\ket{g}$ and excited state $\ket{e}$, and
  \item A \textbf{single electromagnetic mode} of frequency $\omega_c$ confined in an optical or microwave cavity.
\end{itemize}

In this report, we adopt an **Operational** perspective. We do not merely solve the Schrödinger equation; we treat the system as an open quantum system subject to continuous observation (instruments). This aligns the theory with the software architecture of \texttt{jc-sim-oqp}, where "Solvers" correspond to different physical unravellings of the dynamics.

\section{Derivation of the Hamiltonian}
\label{sec:theory}

\subsection{Quantized cavity mode}

We consider a single electromagnetic mode of angular frequency $\omega_c$ in a lossless cavity.
Quantization of this mode leads to bosonic annihilation and creation operators, $\hc$ and $\hcd$, obeying:
\begin{equation}
  [\hc, \hcd] = 1.
\end{equation}
The free Hamiltonian of this mode is (dropping zero-point energy):
\begin{equation}
  \hat{H}_c = \hbar\omega_c \hcd \hc.
\end{equation}
The eigenstates are the Fock states $\{\ket{n}\}_{n=0}^\infty$ satisfying $\hc\ket{n} = \sqrt{n}\ket{n-1}$ and $\hcd\ket{n} = \sqrt{n+1}\ket{n+1}$, with energies $E_n = \hbar\omega_c n$.

\subsection{Two-level atom and Pauli operators}

We approximate the relevant internal structure of the atom (or qubit) by two energy eigenstates, $\ket{g}$ and $\ket{e}$, with energy separation $\hbar\omega_a$.
In this basis one defines the Pauli operators:
\begin{equation}
  \sigmap = \ket{e}\!\bra{g}, \qquad
  \sigmam = \ket{g}\!\bra{e}, \qquad
  \sigmaz = \ket{e}\!\bra{e} - \ket{g}\!\bra{g}.
\end{equation}
The free atomic Hamiltonian is:
\begin{equation}
  \hat{H}_a = \hbar\omega_a \sigmap \sigmam.
\end{equation}

\subsection{Light--matter interaction and RWA}

In the electric-dipole approximation, the interaction between the atom and the quantized field is:
\begin{equation}
  \hat{H}_{\text{int}} = -\hat{\mathbf{d}}\cdot\hat{\mathbf{E}}(\mathbf{r}_a),
\end{equation}
where $\hat{\mathbf{d}} \approx \mathbf{d}_{eg} \sigmap + \mathbf{d}_{ge} \sigmam$.
This leads to a coupling of the form:
\begin{equation}
  \hat{H}_{\text{int}} = \hbar g \left(\hcd + \hc\right)\left(\sigmap + \sigmam\right).
\end{equation}

Moving to the interaction picture, the terms evolve as $e^{\pm i(\omega_c \pm \omega_a)t}$.
Near resonance ($\omega_c \approx \omega_a$), the terms oscillating at $\pm(\omega_c + \omega_a)$ are rapidly varying and average to zero.
We perform the **Rotating Wave Approximation (RWA)** to neglect these terms, yielding:
\begin{equation}
  \hat{H}_{I,\text{RWA}} = \hbar g \left(\hcd \sigmam + \hc \sigmap\right).
\end{equation}

Transforming back to the Schr\"odinger picture, the resulting **Jaynes--Cummings Hamiltonian** is:
\begin{equation}
  \boxed{
  \HJC = \hbar\omega_c \hcd \hc + \hbar\omega_a \sigmap\sigmam
  + \hbar g \left(\hcd \sigmam + \hc \sigmap\right)
  }
  \label{eq:jc_hamiltonian}
\end{equation}

\section{Exact Solution and Dynamics}
\label{sec:exact_solution}

\subsection{Conservation of excitation number}
A key simplification is the conservation of total excitation number:
\begin{equation}
  \hat{N} = \hcd \hc + \sigmap\sigmam.
\end{equation}
One finds $[\HJC, \hat{N}] = 0$. Thus, the Hilbert space decomposes into invariant subspaces labeled by $n$.

\subsection{Dressed states and eigenenergies}
In the basis $\{\ket{n,g}, \ket{n-1,e}\}$ for $n \ge 1$, the Hamiltonian matrix is:
\begin{equation}
  \hat{H}_n = \hbar
  \begin{pmatrix}
    n\omega_c & g\sqrt{n} \\
    g\sqrt{n} & (n-1)\omega_c + \omega_a
  \end{pmatrix}
  = \hbar n\omega_c \mathbb{I} + \hbar
  \begin{pmatrix}
    0 & g\sqrt{n} \\
    g\sqrt{n} & \Delta
  \end{pmatrix}.
\end{equation}
Diagonalizing this matrix yields the eigenenergies:
\begin{equation}
  E_{n,\pm} = \hbar\left(n\omega_c + \frac{\Delta}{2} \pm \frac{\Omega_n}{2}\right),
  \qquad \Omega_n = \sqrt{\Delta^2 + 4g^2 n},
\end{equation}
where $\Delta = \omega_a - \omega_c$ is the detuning.

The eigenvectors, known as **dressed states**, are:
\begin{align}
  \ket{n,+} &= \cos\theta_n\,\ket{n,g} + \sin\theta_n\,\ket{n-1,e}, \\
  \ket{n,-} &= -\sin\theta_n\,\ket{n,g} + \cos\theta_n\,\ket{n-1,e},
\end{align}
where the mixing angle satisfies $\tan(2\theta_n) = 2g\sqrt{n}/\Delta$.

\subsection{Vacuum Rabi oscillations}
On resonance ($\Delta = 0$), $\Omega_n = 2g\sqrt{n}$ and $\theta_n = \pi/4$.
Starting from the state $\ket{0,e}$, the system evolves in the $n=1$ subspace:
\begin{equation}
  \ket{\psi(t)} = \cos(gt)\ket{0,e} - i\sin(gt)\ket{1,g}.
\end{equation}
The atomic excited state probability is:
\begin{equation}
  P_e(t) = |\braket{e}{\psi(t)}|^2 = \cos^2(gt).
\end{equation}
This coherent oscillation at frequency $2g$ is the **Vacuum Rabi Oscillation**.

\subsection{Collapse and revival}
If the cavity is initially in a coherent state $\ket{\alpha} = \sum c_n \ket{n}$, the atom interacts with a superposition of photon numbers.
The probability of excitation involves a sum over many Rabi frequencies:
\begin{equation}
    P_e(t) = \frac{1}{2} + \frac{1}{2} \sum_n |c_n|^2 \cos(2g\sqrt{n}t).
\end{equation}
Because $\sqrt{n} \approx \sqrt{\bar{n}} + \frac{k}{2\sqrt{\bar{n}}}$, the frequencies are incommensurate. This leads to:
1.  **Collapse:** The oscillations dephase on a timescale $t_c \sim 1/g$.
2.  **Revival:** The phases realign at $t_{rev} \sim \frac{2\pi\sqrt{\bar{n}}}{g}$.

%=============================================================================
\part{Foundations: Dynamics of Open Systems}
%=============================================================================

\section{Microscopic Derivation of Dissipation}
\label{sec:open_systems}

While standard texts often postulate the Master Equation, we derive it here from microscopic principles, following the **Reservoir Model** formalism detailed in the main OQP Report.

\subsection{The Reservoir Model}
We model the environment as a bath of thermal harmonic oscillators $\{b_k\}$. The total Hamiltonian is:
\begin{equation}
    H = H_{sys} + \sum_k \omega_k b_k^\dagger b_k + i\hbar \sum_k (\kappa_k b_k^\dagger \hc e^{-i\omega_k t} - \text{h.c.}).
\end{equation}
In the interaction picture, using the **Born Approximation** (weak coupling, $\rho_{tot} \approx \rho \otimes \rho_B$) and the **Markov Approximation** (bath memory time $\tau_B \to 0$ or white noise spectrum), we trace over the bath.

\subsection{The Lindblad Master Equation}
This leads to the **Lindblad Master Equation** for the reduced density matrix $\rho$:
\begin{equation}
  \dv{\hat{\rho}}{t} = -\frac{i}{\hbar}[\HJC, \hat{\rho}] + \sum_k \left( L_k \hat{\rho} L_k^\dagger - \frac{1}{2}\{L_k^\dagger L_k, \hat{\rho}\} \right).
\end{equation}
Our package implements specific collapse operators $L_k$:
\begin{itemize}
    \item \textbf{Cavity Decay:} $L = \sqrt{\kappa}\hc$. Photon loss through mirrors.
    \item \textbf{Thermal Excitation:} $L = \sqrt{\kappa \bar{n}_{th}}\hcd$. Blackbody radiation.
    \item \textbf{Spontaneous Emission:} $L = \sqrt{\gamma}\sigmam$. Atomic decay to free space.
    \item \textbf{Pure Dephasing:} $L = \sqrt{\gamma_\phi}\sigmaz$. Elastic scattering.
\end{itemize}

\section{Stochastic Unravellings and Minimal Dilation}
\label{sec:dilation}

\subsection{The Problem of Dimensionality}
The density matrix $\rho$ has dimension $N^2$. For $N=100$, this is $10^4$ variables. For multi-atom systems, it becomes $2^{2M}$.
Solving the Master Equation directly ($\mathcal{O}(D^3)$ to $\mathcal{O}(D^6)$) is intractable.

\subsection{Minimal Dilation Theorem}
We rely on the **Minimal Dilation Theorem** (Maassen, Sz.-Nagy), a core result of OQP.
\begin{theorem}[Minimal Dilation]
Let $C_t$ be a contractive semigroup on a Hilbert space $\mathcal{H}$. There exists a larger space $\mathcal{K} \supset \mathcal{H}$ and a unitary group $U_t$ on $\mathcal{K}$ such that $C_t(\rho) = \Tr_{aux}(U_t (\rho \otimes \rho_{vac}) U_t^\dagger)$.
\end{theorem}
Physically, this means we can "unravel" the mixed state evolution into pure state trajectories on a conditional space.

\subsection{Monte Carlo Wavefunctions (Quantum Trajectories)}
This theoretical insight leads directly to the **Stochastic Solver**.
We continuously monitor the bath. This yields a measurement record $N(t)$ (e.g. photon clicks).
The conditional state $\ket{\psi_c(t)}$ evolves under the **Stochastic Schrödinger Equation**:
1.  **Drift:** Between jumps, evolution is governed by a non-Hermitian Hamiltonian:
    \begin{equation}
        H_{eff} = H - \frac{i\hbar}{2}\sum_k L_k^\dagger L_k.
    \end{equation}
    (This causes the norm to decay, reflecting the "waiting time" distribution).
2.  **Jump:** When a jump occurs (probability $\delta p = \bra{\psi}L^\dagger L \ket{\psi}dt$), the state collapses:
    \begin{equation}
        \ket{\psi} \to \frac{L_k \ket{\psi}}{||L_k \psi||}.
    \end{equation}

The ensemble average of these trajectories recovers the exact master equation:
\begin{equation}
    \rho(t) = \frac{1}{M} \sum_{i=1}^M \ket{\psi_i(t)}\bra{\psi_i(t)}.
\end{equation}
This method scales as $\mathcal{O}(N)$, enabling simulations of much larger systems.

%=============================================================================
\part{Information and Entropy: Quantifying Correlations}
%=============================================================================

\section{Evolution of Quantum Entanglement}
\label{sec:entanglement}

\subsection{Introduction}
The link between quantum physics and information theory is forged through the concepts of \emph{entropy} and \emph{information} (also called \emph{negentropy}). In quantum physics, negentropy is seen as a measure of coherence, non-separability, or \emph{quantum entanglement}.
Entanglement of two systems is a purely quantum phenomenon, by which, following an interaction, the two quantum systems maintain a coherence between them, even if the interaction ceases.

\subsection{Entropy and Entanglement}

\subsubsection{The Quantum Case (Pure States)}

In the quantum case, classical definitions are insufficient. Quantum states are described by trace-class operators on a Hilbert space. Composite systems exhibit entanglement. Mathematically, entanglement means the state operator $\hat{\rho}$ in $\mathcal{H} = \mathcal{H}' \otimes \mathcal{H}''$ cannot be written as a product state.
If the systems are not separated, even if the composite system is in a pure state, the state vector can be brought to the Schmidt bi-orthogonal form:
\begin{equation}
    |\Psi\rangle = \sum_s \sqrt{\lambda_s} |u_s\rangle |v_s\rangle
\end{equation}
where $\{\lambda_i\}$ are the eigenvalues of the reduced density operator $\hat{\rho}' = \Tr'' |\Psi\rangle\langle\Psi|$.

The quantitative measure of entanglement for pure states is the Von Neumann entropy of the reduced density matrix:
\begin{equation}
    E(\Psi) = S(\hat{\rho}') = -\Tr(\hat{\rho}' \ln \hat{\rho}') = - \sum_s \lambda_s \ln \lambda_s
\end{equation}

\subsubsection{Entanglement Measures for Mixed States}
For mixed states, the entropy of the reduced subsystem is no longer a good measure of entanglement. Instead, we define entanglement measures based on convex roof constructions or distance measures.
The \textbf{Relative Entropy of Entanglement} is defined as the minimum "distance" to the set of separable states $\mathcal{S}$:
\begin{equation}
    E_{rel}(\hat{\rho}) = \min_{\hat{\sigma} \in \mathcal{S}} S(\hat{\rho} || \hat{\sigma})
\end{equation}

And the \textbf{Renyi Entropy} of order $\alpha$ generalizes the Shannon form:
\begin{equation}
    S_\alpha(\hat{\rho}) = \frac{1}{1-\alpha} \ln \Tr(\hat{\rho}^\alpha)
\end{equation}
For $\alpha=2$, this measures the \textbf{Purity} of the state, a key metric in experimental characterization.

%=============================================================================
\part{Operational Measurement Theory}
%=============================================================================

\section{The Dispersive Regime}
\label{sec:dispersive}

\subsection{Effective Hamiltonian}
When the detuning is large ($|\Delta| \gg g\sqrt{n}$), energy exchange is suppressed.
A second-order Schrieffer--Wolff transformation eliminates the interaction term to first order, yielding:
\begin{equation}
  \Hdisp \approx \hbar\omega_c \hcd \hc + \frac{\hbar}{2}(\omega_a + \chi)\sigmaz + \hbar\chi\,\hcd\hc\,\sigmaz,
\end{equation}
where $\chi = g^2/\Delta$ is the dispersive shift.
This Hamiltonian is diagonal in the bare basis. It explains two key effects:
1.  **AC Stark Shift:** The qubit frequency shifts by $2\chi n$.
2.  **Cavity Pull:** The cavity frequency shifts by $\pm \chi$ depending on the qubit state $\sigma_z$.

This is the standard regime for **Circuit QED Readout**.

\section{Quantum Instruments and Tomography}
\label{sec:instruments}

\subsection{Measurement as an Instrument}
In OQP, a measurement is defined by a **Quantum Instrument** $\mathcal{I}$.
An instrument is a collection of completely positive maps $\{\mathcal{E}_x\}$ summing to a trace-preserving map.
For dispersive readout, the instrument correlates the pointer (cavity phase) with the system (qubit).

\subsection{Fuzzification and POVMs}
Realistic measurements are never ideal projectors. As derived in the Report's Stern-Gerlach analysis, physical noise (gradient fluctuations, amplifier noise) leads to **Fuzzification**.
The observable is derived from a **Positive Operator-Valued Measure (POVM)**:
\begin{equation}
    F_\alpha = \int d\mu(x) \omega(x|\alpha) P_x
\end{equation}
where $\omega$ is a confusion kernel (e.g. Gaussian overlap of coherent states).
The \texttt{jc-sim-oqp} package acknowledges this by distinguishing between the *internal* state and the *readout* result.

\section{Dynamic Models of Quantum Measurements: The Stern-Gerlach Example}
\label{sec:dynamic_models}

\subsection{The Pauli-Schrödinger Equation}

We consider a neutral particle with spin 1/2 and mass $m$ moving in a magnetic field $\vec{B}$. Unlike the impulsive approximation used in OQP (where kinetic energy is neglected), here we solve the full **Pauli-Schrödinger equation**:
\begin{equation}
    i\hbar \pdv{\Psi}{t} + V(\hat{\vec{r}})\Psi = 0
\end{equation}
where the potential energy operator is given by the interaction of the magnetic moment with the field:
\begin{equation}
    V(\hat{\vec{r}}) = - \hat{\vec{\mu}} \cdot \vec{B}(\hat{\vec{r}}) = - \mu_0 \hat{\vec{\sigma}} \cdot \vec{B}(\hat{\vec{r}})
\end{equation}
Assuming a magnetic field with a strong uniform component $B_0$ along the $z$-axis and a gradient $b$, we can neglect the transverse non-uniform components (justified by the interaction picture for sufficiently large $B_0$). The field is approximated as:
\begin{equation}
    \vec{B} \approx (B_0 + bz) \vec{e}_z
\end{equation}

In the momentum representation ($p_z = \hbar k$), the equation becomes a first-order PDE solvable by the method of characteristics. The solution corresponds to separating wavepackets:
\begin{equation}
    \psi_\pm(\zeta, \tau) = \frac{1}{\pi^{1/4}\zeta_0^{1/2}\sqrt{1+i\tau/\zeta_0^2}} \exp\left( - \frac{(\zeta \mp \tau^2)^2}{2\zeta_0^2(1+i\tau/\zeta_0^2)} \right)
\end{equation}

\subsection{Stochastic Perturbation Model: Microscopic Origin of Noise}

The ideal separation derived above assumes a perfectly isolated system. OQP Chapter 7 explicitly models the environment using the **Quantum Noise** formalism.
We modify the Hamiltonian by adding a stochastic perturbation term $\hat{B}(t)$, corresponding to interaction with a thermal reservoir.
\begin{equation}
    i\hbar \pdv{\psi_\pm}{t} \mp \mu_0 b \hat{z} \psi_\pm + \hat{B}(t) \hat{z} \psi_\pm = 0
\end{equation}
In the Markovian limit, this interaction leads to a **Quantum Dynamical Semigroup** generator $\mathcal{L}$ of the **Lindblad form**, physically corresponding to momentum diffusion. This diffusion broadens the wave packets in momentum space, creating an overlap region where the spin components are not perfectly distinguishable.

This overlap is the physical origin of the "fuzziness" of the measurement. It provides the constructive proof for the existence of the specific Instrument used in OQP.

\subsection{Covariant Tomography Protocol}
Given that the real SG device implements a fuzzy POVM, how do we characterize it experimentally? We proposed a protocol we called "Covariant Tomography" (a precursor to Gate Set Tomography).
The probability of detection for a fixed input state $\rho_{in}$ is:
\begin{equation}
    f(\theta, \phi) = \Tr(F(\theta, \phi) \rho_{in})
\end{equation}
Expanding this, we found a characteristic harmonic signature:
\begin{equation}
    f(\theta, \phi) = c_0 + c_1 \sin\theta \cos\phi + c_2 \sin\theta \sin\phi + c_3 \cos\theta
\end{equation}
The coefficients $c_i$ depend linearly on the unknown instrument parameters $\alpha, \vec{\beta}$. Solving the linear system yields the full characterization of the non-ideal device.

%=============================================================================
\part{Appendix: Classical Physics Formalism}
%=============================================================================

\section{Symplectic Manifolds and Liouville Dynamics}
\label{app:classical}

This appendix provides detailed mathematical background on classical physics concepts referenced in the main text.

\subsection{Deterministic Evolution on Symplectic Manifolds}

In classical mechanics, the states of a controllable system are described by points $x$ belonging to a symplectic differentiable manifold $(\mathbb{X}, \sigma)$. $\sigma(\bullet, \bullet)$ is a real, bilinear, antisymmetric and maximally closed form.
The evolution of a system interacting with a force field ensuring the conservation of energy $H$ (conservative) is given by the vector field:
\begin{equation}
    \dot{x} = \mathcal{I} dH(x)
\end{equation}
The solution for this equation can be written with the help of a one-parameter group of diffeomorphisms $g^t: \mathbb{X} \to \mathbb{X}$:
\begin{equation}
    x(t) = g^t x(0)
\end{equation}
called the Hamiltonian flow of the function $H$.

\subsection{Stochastic Evolution and Martingales}

If the classical system is not fully controllable, the total energy can no longer be conserved.
Let a probability space $(\Omega, \mathcal{F}, \mu)$ be defined. A family of random variables $f_t$ defined on the probability space is called a \emph{martingale} if $f_t$ is $\mathcal{F}_t$-measurable for any $t$ and $s \le t \implies M(f_t | \mathcal{F}_s) = f_s$.
The physical interpretation of filtrations is that of the set of information accumulated up to time $t$ (history).

%=============================================================================
\part{Experimental Considerations}
%=============================================================================

\section{Experimental Realizations}
\label{sec:experiments}

\subsection{Optical cavity QED with neutral atoms}
In optical cavity QED, neutral atoms (e.g. Rb, Cs) are trapped in high-finesse cavities.
\begin{itemize}
    \item **Parameters:** $\omega_c \sim 300$ THz, $g \sim 10$ MHz.
    \item **Regime:** Strong coupling is achievable ($g \gg \kappa, \gamma$).
    \item **Relevance:** Foundation for quantum networks and distributed quantum computing.
\end{itemize}

\subsection{Superconducting Circuit QED}
In circuit QED, Transmon qubits couple to coplanar waveguide resonators.
\begin{itemize}
    \item **Parameters:** $\omega_c \sim 5$ GHz, $g \sim 100$ MHz.
    \item **Regime:** Can reach Ultrastrong coupling.
    \item **Features:** Lithographically engineerable. Dispersive readout is the standard.
\end{itemize}

\section{Numerical Simulation Guidelines}
\label{sec:numerics}

\subsection{Hilbert-space execution}
The cavity Fock space is infinite and must be truncated.
For coherent state simulations with mean photon number $\bar{n}$, we recommend:
\begin{equation}
  N_{trunc} \approx \bar{n} + 5\sqrt{\bar{n}}.
\end{equation}
This ensures neglecting only exponentially small amplitudes.

\subsection{Solver Choice}
\begin{table}[h]
  \centering
  \begin{tabular}{llp{6cm}}
    \toprule
    \textbf{Solver} & \textbf{Method} & \textbf{Best Use Case} \\
    \midrule
    \texttt{ExactSolver} & Master Eq (mesolve) & General dynamics, small N, accurate density matrix. \\
    \texttt{StochasticSolver} & Trajectories (mcsolve) & Large N systems, single-shot experiment simulation, entanglement tracking. \\
    \texttt{DispersiveSolver} & Effective H & Fast readout simulation, large detuning limit ($|\Delta| \gg g$). \\
    \bottomrule
  \end{tabular}
\end{table}

\end{document}

